% !TEX root = ../my-thesis.tex
%
\pdfbookmark[0]{Abstract}{Abstract}
\chapter*{Abstract}
\label{sec:abstract}
\vspace*{-10mm}

Topic modeling is a popular approach to study large document collections. It returns topics as a set of coherent words, that are usually manually labeled, regarding the concept that the top 10 words of a topic are describing. In this work is examined how the labeling of topics can be automated. Further, the internal consistency  is studied when the number of topics increases. Both approaches shall help domain experts using topic modeling for their work and understand which effect a higher or low number of topics has on the quality of the generated topics. To evaluate the results multiple datasets of online discussions regarding organic products are analyzed.

\vspace*{20mm}

{\usekomafont{chapter} Zusammenfassung}\label{sec:abstract-diff} \\

Topic modeling ist eine beliebte Methode große Textsammlung zu analysieren. Üblicherweise werden die identifizierten Themen als Liste an semantisch zusammenhängenden Wörtern dagestellt. Dieser Wörterliste wird dann manuell ein Betreff hinzugefügt. In dieser Arbeit werden unterschiedliche Ansätze verglichen, um diesen Prozess zu automatisieren. Des Weiteren wird die interne Konsistenz der Modelle bei der Veränderung der Topicanzahl analysiert. Beide Analysen sollen es Domänenexperten in Zukunft leichter machen Topic Modeling für ihre Arbeit einzusetzen. Außerdem, wird den Fachexperten dadurch ein besseres Verständnis für den Effekt der festzulegenden Topicanzahl vermittelt. Um die Ergebnisse zu überprüfen, werden die Analysen auf mehreren Datensätzen zu Onlinediskussionen über Bio-Produkten durchgeführt.
\newpage