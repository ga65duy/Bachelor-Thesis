\chapter{Future Work and Conclusion}
\label{future_work}

\section{Future work}
- Mai bessere Kl oder andere funktion generieren, sodass die labels besser zu den topics passen

- haben topic labeling mit ontology nur auf englischsprachigen topics angewandt, man könnte versuchen, dies  auch mit Thesaurus db auf den deutschen topics anzupassen.

- ein verfahren für topic labeling, dass man unabhängig von der sprache bzw vom algorithmus lda bzw nmf und weitere, verwenden kann. 

Intern consistency: compare topic models based on changing other parameters such as alpha, beta, for lda, nmf only topic number as übergabeparameter

\section{Conclusion}
In this thesis two main topics were covered: How can we label topics automatically and how can we measure the internal consistency of topic modeling. 

To answer the fist question the intrinsic and extrinsic automatic topic labeling was introduced an evaluated on English and German editorial articles. The evaluation showed, that the intrinsic approach produced meaningful labels on their own, but theses did not fit to the topics. The extrinsic approach was more successful,
On average with this method labels were generated, which were fitting more to the topics and some of the automatic labels even matched with the labels, that were submitted by the domain experts. 

To answer the second question, several key figures were introduced to measure the internal consistency of topic models with a different number of topic numbers. All key figures were applied on our dataset. 

...
No of the key figures is able to answer the question. Further we also did correlations....
